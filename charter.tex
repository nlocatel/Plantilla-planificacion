\documentclass[11pt]{charter}

\usepackage{pdflscape}
\usepackage{tikz}
\usetikzlibrary{positioning}
\usepackage{pgfgantt}
\usepackage{comment}

% El títulos de la memoria, se usa en la carátula y se puede usar el cualquier lugar del documento con el comando \ttitle
\titulo{Aplicación web para sistema de homologación de relés de señalamiento ferroviario tipo B1}

% Nombre del posgrado, se usa en la carátula y se puede usar el cualquier lugar del documento con el comando \degreename
\posgrado{Carrera de Especialización en Sistemas Embebidos} 
%\posgrado{Carrera de Especialización en Internet de las Cosas} 
%\posgrado{Carrera de Especialización en Intelegencia Artificial}
%\posgrado{Maestría en Sistemas Embebidos} 
%\posgrado{Maestría en Internet de las cosas}

% Tu nombre, se puede usar el cualquier lugar del documento con el comando \authorname
\autor{Ing. Nicolás Locatelli} 

% El nombre del director y co-director, se puede usar el cualquier lugar del documento con el comando \supname y \cosupname y \pertesupname y \pertecosupname
\director{Ing. Gustavo Ramoscelli}
\pertenenciaDirector{UNS,CONICET-GICSAFe} 
% FIXME:NO IMPLEMENTADO EL CODIRECTOR ni su pertenencia
\codirector{} % si queda vacio no se deberíá incluir 
\pertenenciaCoDirector{}

% Nombre del cliente, quien va a aprobar los resultados del proyecto, se puede usar con el comando \clientename y \empclientename
\cliente{Ing. Martín Harris}
\empresaCliente{Trenes Argentinos}

% Nombre y pertenencia de los jurados, se pueden usar el cualquier lugar del documento con el comando \jurunoname, \jurdosname y \jurtresname y \perteunoname, \pertedosname y \pertetresname.
\juradoUno{Nombre y Apellido (1)}
\pertenenciaJurUno{pertenencia (1)} 
\juradoDos{Nombre y Apellido (2)}
\pertenenciaJurDos{pertenencia (2)}
\juradoTres{Nombre y Apellido (3)}
\pertenenciaJurTres{pertenencia (3)}
 
\fechaINICIO{22 de junio de 2020}		%Fecha de inicio de la cursada de GdP \fechaInicioName
\fechaFINALPlanificacion{22 de Agosto de 2020} 	%Fecha de final de cursada de GdP
\fechaFINALTrabajo{22 de diciembre de 2020}		%Fecha de defensa pública del trabajo final


\begin{document}

\maketitle
\thispagestyle{empty}
\pagebreak


\thispagestyle{empty}
{\setlength{\parskip}{0pt}
\tableofcontents{}
}
\pagebreak


\section{Registros de cambios}
\label{sec:registro}


\begin{table}[ht]
\label{tab:registro}
\centering
\begin{tabularx}{\linewidth}{@{}|c|X|c|@{}}
\hline
\rowcolor[HTML]{C0C0C0} 
Revisión & \multicolumn{1}{c|}{\cellcolor[HTML]{C0C0C0}Detalles de los cambios realizados} & Fecha      \\ \hline
1.0      & Creación del documento                                          & 27/06/2020 \\ \hline
1.1      & Avance sobre puntos 1 a 6 del documento                                                               & 09/07/2020 \\ \hline
1.2      & Avance sobre puntos 7 a 12 del documento                                                               & 27/07/2020 \\ \hline
1.3      & Avance sobre puntos 13 a 17 del documento                                                               & 05/08/2020 \\ \hline
1.3.1    & Modificaciones por sugerencias del cuerpo Directivo                                                               & 10/08/2020 \\ \hline
1.4      & Agrego historias de usuarios y correcciones                                                               & 14/08/2020 \\ \hline
\end{tabularx}
\end{table}

\pagebreak



\section{Acta de Constitución del Proyecto}
\label{sec:acta}

\begin{flushright}
Buenos Aires, \fechaInicioName
\end{flushright}

\vspace{2cm}

Por medio de la presente se acuerda con el Ing. \authorname\hspace{1px} que su Trabajo Final de la \degreename\hspace{1px} se titulará ``\ttitle'', consistirá esencialmente en el prototipo preliminar de un sitio web para la configuración y muestra de resultados de los ensayos a realizar por el sistema probador de relés ferroviarios.
Tendrá un presupuesto preliminar estimado de 600 hs de trabajo, con fecha de inicio \fechaInicioName\hspace{1px} y fecha de presentación pública \fechaFinalName.

Se adjunta a esta acta la planificación inicial.

\vfill

% Esta parte se construye sola con la información que hayan cargado en el preámbulo del documento y no debe modificarla
\begin{table}[ht]
\centering
\begin{tabular}{ccc}
\begin{tabular}[c]{@{}c@{}}Ing. Ariel Lutenberg \\ Director posgrado FIUBA\end{tabular} &  & \begin{tabular}[c]{@{}c@{}}\clientename \\ \empclientename \end{tabular} \vspace{2.5cm} \\ 
\multicolumn{3}{c}{\begin{tabular}[c]{@{}c@{}} \supname \\ Director del Trabajo Final\end{tabular}} \vspace{2.5cm} \\
\begin{tabular}[c]{@{}c@{}}\jurunoname \\ Jurado del Trabajo Final\end{tabular}     &  & \begin{tabular}[c]{@{}c@{}}\jurdosname\\ Jurado del Trabajo Final\end{tabular}  \vspace{2.5cm}  \\
\multicolumn{3}{c}{\begin{tabular}[c]{@{}c@{}} \jurtresname\\ Jurado del Trabajo Final\end{tabular}} \vspace{.5cm}                                                                     
\end{tabular}
\end{table}


\section{Descripción técnica-conceptual del Proyecto a realizar}
\label{sec:descripcion}

La organización CONICET-GICSAFe se caracteriza por su misión de realizar proyectos ferroviarios basados en electrónica e informática con alto impacto social y económico.

En el marco del proyecto ``\ttitle'', este subproyecto tiene por finalidad la realización de una interfaz de usuario web, que permita al personal encargado de los ensayos aplicados a los relés ferroviarios, configurar fácilmente dichos ensayos y poder visualizar de forma resumida y clara los resultados de los mismos.

El presente proyecto se destaca  especialmente por ser el primero de su tipo en Argentina. No hay hasta el momento un emprendimiento destinado a la fabricación y prueba local de relés de tipo ferroviario.

En la Figura \ref{fig:diagBloques} se presenta el diagrama conceptual del sistema. El bloque resaltado en azul es la parte correspondiente al proyecto descripto en este documento. Se muestra la relación que tiene con el resto de las partes del sistema.

\vspace{25px}

\begin{figure}[htpb]
\centering 
\includegraphics[width=.7\textwidth]{./Figuras/diagramaConceptual.png}
\caption{Diagrama conceptual del sistema}
\label{fig:diagBloques}
\end{figure}

\vspace{25px}

\newpage

\section{Identificación y análisis de los interesados}
\label{sec:interesados}

\begin{table}[ht]
%\caption{Identificación de los interesados}
%\label{tab:interesados}
\begin{tabularx}{\linewidth}{@{}|l|X|X|l|@{}}
\hline
\rowcolor[HTML]{C0C0C0} 
Rol           & Nombre y Apellido & Organización 	& Puesto 	\\ \hline
Cliente       & \clientename      &\empclientename	& -      	\\ \hline
Impulsor      & Ing. Mariano Fernandez Soler  & Trenes Argentinos   	& -        	\\ \hline
Responsable   & \authorname       & CONICET-GICSAFe        	& Alumno 	\\ \hline
Orientador    & \supname	      & \pertesupname 	& Director Trabajo final \\ \hline
Colaborador    & Ing. Ariel Lutenberg & CONICET-GICSAFe 	& Director Posgrado \\ \hline
Equipo        & Ing. Gustavo Ramoscelli & \pertesupname  	& Docente        	\\ \hline
Usuario final & Operador del laboratorio de ensayos &\empclientename 	& -       	\\ \hline
\end{tabularx}
\end{table}


\section{1. Propósito del proyecto}
\label{sec:proposito}

El propósito de este proyecto es implementar un sitio web para facilitar la configuración de los ensayos a realizar a cada relé y la representación gráfica de los
datos que surgen como resultado de los mismos.

\section{2. Alcance del proyecto}
\label{sec:alcance}

El presente proyecto incluye:
\begin{itemize}
\item Autenticación de usuarios mediante nombre de usuario y contraseña.
\item Esquema de autorización de usuarios mediante tres roles diferentes: administrador, configurador y usuario de sólo lectura.
\item Vista de configuración para los distintos tipos de ensayos (1, 2 y 3).
\item Vista de los resultados de cada ensayo realizado.
\item Persistencia de configuraciones y resultados en base de datos PostGres provista por el cliente.
\end{itemize}

El presente proyecto NO incluye:
\begin{itemize}
\item Nada por fuera de lo mencionado en el alcance.
\end{itemize}


\section{3. Supuestos del proyecto}
\label{sec:supuestos}

Para el desarrollo del presente proyecto se supone:

\begin{itemize}
\item Los requerimientos no sufrirán modificaciones de consideración durante la implementación del proyecto.
\item Disponer de los recursos necesarios (PC, acceso a internet, software utilizado) para realizar la tarea.
\item El cliente proveerá el hardware sobre el cuál se instalará el sitio web y la base de datos.
\end{itemize}


\section{4. Requerimientos}
\label{sec:requerimientos}

\begin{enumerate}

\item Requerimientos relativos a usuarios:
	\begin{enumerate}
	\item Se pedirá al usuario autenticarse mediante nombre y password para usar el sistema. El usuario no autenticado será redirigido a una pantalla de login.
	\item Habrá tres (3) roles de usuarios: Administrador, Configurador y Usuario de solo lectura.
		\begin{enumerate}
		\item El usuario de tipo 'Administrador' sólo será autorizado a realizar operaciones relativas a administrar usuarios.
		\item El usuario de tipo 'Configurador' sólo será autorizado a realizar operaciones relativas a configuración de ensayos.
		\item El usuario de tipo 'Solo lectura' sólo será autorizado a realizar operaciones relativas a la visualización de resultados.
		\end{enumerate}
	\end{enumerate}

\item Requerimientos relativos configuración de ensayos:
	\begin{enumerate}
	\item Habrá una vista para la configuración de ensayo. La vista permitirá al usuario seleccionar el relé, indicar el tipo de ensayo y los parámetros pertinentes al ensayo.
	\item Los datos indicados para el ensayo serán validados, con indicación visual y clara para el usuario.
	\end{enumerate}

\item Requerimientos relativos a visualización de ensayos:
	\begin{enumerate}
	\item Habrá una vista del listado índice de relés.
	\item Al seleccionar un relé del listado, se mostrará una vista de resultados de ensayos del relé seleccionado.
	\item La vista de resultados de ensayos del relé mostrará los resultados de los ensayos en forma numérica y gráfica, de fácil interpretación por parte del usuario.
	\end{enumerate}

\item Requerimientos relativos a la comunicación con el hardware:
	\begin{enumerate}
	\item El sistema deberá ser capaz de comunicarse con potencialmente más de un equipo probador de relés. Inicialmente será un único equipo, pero este número puede ser mayor en el futuro.
	\end{enumerate}

\item Requerimientos relativos a la persistencia de datos:
	\begin{enumerate}
	\item Los datos serán persistidos en un servidor y base de datos provista por el cliente.
	\item Cualquier registro de la base de datos, tendrá referencia al último usuario que lo modificó, junto con la fecha.
	\end{enumerate}

	
\end{enumerate}

\section{5. Entregables principales del proyecto}
\label{sec:entregables}

\begin{itemize}
\item Manual de usuario
\item Código fuente
\item Informe final
\end{itemize}


\section{Historias de usuarios (product Backlog)}

Se toma como criterio de prioridad un número entero entre 1 y 10 que representa la necesidad de la característica. Se toma como criterio de ponderación un número entero entre 1 y 10 que representa el tiempo de trabajo para implementar la característica.

\begin{itemize}

\item Como usuario, quiero que las pantallas sean sencillas, sin detalles innecesarios, para interpretar fácilmente la información.
	\begin{itemize}
	\item Prioridad: 9
	\item Ponderación: 6
	\end{itemize}

\item Como operador, quiero que los datos que ingreso no se pierdan si por error salgo de una pantalla, para no tener que volver a ingresarlos.
	\begin{itemize}
	\item Prioridad: 7
	\item Ponderación: 8
	\end{itemize}

\item Como cliente, quiero poder ver los resultados de los ensayos desde mi celular, para poder verlos desde cualquier lugar.
	\begin{itemize}
	\item Prioridad: 10
	\item Ponderación: 6
	\end{itemize}

\item Como cliente, quiero que me llegue un mail cuando un ensayo se completa, con un resumen del resultado.
	\begin{itemize}
	\item Prioridad: 8
	\item Ponderación: 6
	\end{itemize}

\item Como cliente, quiero un panel donde pueda ver una estadística general de los ensayos, con varios criterios de selección.
	\begin{itemize}
	\item Prioridad: 8
	\item Ponderación: 10
	\end{itemize}

\end{itemize}


\section{6. Desglose del trabajo en tareas}
\label{sec:wbs}

\begin{enumerate}
\item Grupo de tareas de planificación
	\begin{enumerate}
	\item Leer documentación del proyecto. (48 hs)
	\item Reuniones con el equipo. (16 hs)
	\end{enumerate}

\item Grupo de tareas de preparación
	\begin{enumerate}
	\item Organizar herramientas de desarrollo. (16 hs)
	\item Generación del entorno y el repositorio. (16 hs)
	\end{enumerate}

\item Grupo de tareas relacionadas con el modelo
	\begin{enumerate}
	\item Definir entidades y sus relaciones. (16 hs)
	\item Crear las migraciones para generar las tablas. (8 hs)
	\item Crear seeders para las tablas. (8 hs)
	\item Crear servicio de acceso a base de datos. (16 hs)
	\item Crear API de acceso a datos. (16 hs)
	\item Verificación. (8 hs)
	\end{enumerate}

\item Grupo de tareas relacionadas con usuarios
	\begin{enumerate}
	\item Estudiar y agregar plugin de autenticación. (8 hs)
	\item Estudiar plugin de roles. (16 hs)
	\item Agregar plugin de roles y definir roles. (16 hs)
	\item Verificación. (8 hs)
	\end{enumerate}

\item Grupo de tareas relacionadas con el servidor Nodered
	\begin{enumerate}
	\item Implementar recepción e inserción de registros. (24 hs)
	\end{enumerate}

\item Grupo de tareas relacionadas con configuración de ensayos
	\begin{enumerate}
	\item Definir vistas de configuración. (16 hs)
	\item Implementar componentes Vue. (24 hs)
	\item Implementar vistas. (24 hs)
	\item Verificación. (8 hs)
	\end{enumerate}

\item Grupo de tareas relacionadas con visualización de ensayos
	\begin{enumerate}
	\item Estudiar Google Vue Charts. (8 hs)
	\item Diseñar vistas para cada tipo de ensayo. (16 hs)
	\item Implementar componente Vue para cada tipo. (48 hs)
	\item Verificación. (8 hs)
	\end{enumerate}

\item Grupo de tareas relacionadas con el area de hardware
	\begin{enumerate}
	\item Acordar estructuras de datos. (16 hs)
	\item Reuniones de asistencia e intercambio. (16 hs)
	\end{enumerate}

\item Grupo de tareas relacionadas con la aprobación
	\begin{enumerate}
	\item Realizar los tests de aprobación. (32 hs)
	\end{enumerate}

\end{enumerate}

Cantidad total de horas: (456 hs)



\begin{landscape}

\begin{figure}[htb]
\section{7. Diagrama de Activity On Node}
\label{sec:AoN}

    \centering
    \resizebox{1.5\textwidth}{!}{%
\begin{tikzpicture}[
hito/.style={circle, draw=black, thick, minimum size=2cm},
tarea/.style={rectangle, draw=black, thick, minimum size=2cm, text width=2em, text centered},
]
%Nodes
\node[hito]      (inicio)                              	{Inicio};
\node[tarea]     (tarea_1_1)       [right=of inicio] 	{Tarea 1.1\\t=6d};
\node[tarea]     (tarea_1_2)       [below=of tarea_1_1] {Tarea 1.2\\t=2d};

\node[tarea]     (tarea_2_1)       [right=of tarea_1_2] {Tarea 2.1\\t=2d};
\node[tarea]     (tarea_2_2)       [right=of tarea_2_1] {Tarea 2.2\\t=2d};

\node[hito]      (empezar)		   [right=of tarea_2_2] {Listos};

\node[tarea]     (tarea_3_1)       [right=of empezar] 	{Tarea 3.1\\t=2d};
\node[tarea]     (tarea_3_2)       [right=of tarea_3_1] {Tarea 3.2\\t=1d};
\node[tarea]     (tarea_3_3)       [right=of tarea_3_2] 	{Tarea 3.3\\t=1d};
\node[tarea]     (tarea_3_4)       [below=of tarea_3_3] 	{Tarea 3.4\\t=2d};
\node[tarea]     (tarea_3_5)       [below=of tarea_3_4] 	{Tarea 3.5\\t=2d};
\node[tarea]     (tarea_3_6)       [right=of tarea_3_5] 	{Tarea 3.6\\t=1d};

\node[hito]      (modelo_listo)	   [right=of tarea_3_6] {Modelo listo};

\node[tarea]     (tarea_4_1)       [right=of modelo_listo] 	{Tarea 4.1\\t=1d};
\node[tarea]     (tarea_4_2)       [below=of tarea_4_1] 	{Tarea 4.2\\t=2d};
\node[tarea]     (tarea_4_3)       [right=of tarea_4_2] 	{Tarea 4.3\\t=2d};
\node[tarea]     (tarea_4_4)       [right=of tarea_4_3] 	{Tarea 4.4\\t=1d};

\node[tarea]     (tarea_5_1)       [below=of tarea_4_2] 	{Tarea 5.1\\t=3d};

\node[tarea]     (tarea_6_1)       [below=of tarea_5_1] 	{Tarea 6.1\\t=2d};
\node[tarea]     (tarea_6_2)       [right=of tarea_6_1] 	{Tarea 6.2\\t=3d};
\node[tarea]     (tarea_6_3)       [below=of tarea_6_2] 	{Tarea 6.3\\t=3d};
\node[tarea]     (tarea_6_4)       [right=of tarea_6_3] 	{Tarea 6.4\\t=1d};

\node[tarea]     (tarea_7_1)       [below=of tarea_6_3] 	{Tarea 7.1\\t=1d};
\node[tarea]     (tarea_7_2)       [right=of tarea_7_1] 	{Tarea 7.2\\t=2d};
\node[tarea]     (tarea_7_3)       [below=of tarea_7_2] 	{Tarea 7.3\\t=6d};
\node[tarea]     (tarea_7_4)       [right=of tarea_7_3] 	{Tarea 7.4\\t=1d};

\node[tarea]     (tarea_8_1)       [below=of tarea_3_1] 	{Tarea 8.1\\t=2d};
\node[tarea]     (tarea_8_2)       [right=of tarea_8_1] 	{Tarea 8.2\\t=2d};

\node[hito]      (implementado)	   [right=of tarea_7_4] {Implementado};

\node[tarea]     (tarea_9_1)       [right=of implementado] 	{Tarea 9.1\\t=4d};

\node[hito]      (fin)	   [right=of tarea_9_1] {Fin};

%Lines
%\draw [->] (inicio) -- (tarea_1_1);
\end{tikzpicture}

}%
  \caption{Diagrama Activity On Node}
  \end{figure}

\newpage

\begin{figure}[htb]
\section{8. Diagrama de Gantt}
\label{sec:gantt}
    \centering
    \resizebox{1.5\textwidth}{!}{%

\begin{ganttchart}[
canvas/.append style={fill=none, draw=black!5, line width=.75pt},
hgrid style/.style={draw=black!5, line width=.75pt},
vgrid={*1{draw=black!5, line width=.75pt}},
today=1,
today rule/.style={
draw=black!64,
dash pattern=on 3.5pt off 4.5pt,
line width=1.5pt
},
today label font=\small\bfseries,
title/.style={draw=none, fill=none},
title label font=\bfseries\footnotesize,
title label node/.append style={below=7pt},
include title in canvas=false,
bar label font=\mdseries\small\color{black!70},
bar label node/.append style={left=2cm},
bar/.append style={draw=none, fill=black!63},
bar incomplete/.append style={fill=grey},
bar progress label font=\mdseries\footnotesize\color{black!70},
group incomplete/.append style={fill=blue},
group left shift=0,
group right shift=0,
group height=.3,
group peaks tip position=0,
group label node/.append style={left=.6cm},
group progress label font=\bfseries\small,
link/.style={-latex, line width=1pt, red},
link label font=\scriptsize\bfseries,
link label node/.append style={below left=-2pt and 0pt},
]{1}{132}

\gantttitle{\ttitle}{132} \\[grid]
\gantttitle{Agosto}{30}
\gantttitle{Septiembre}{30}
\gantttitle{Octubre}{30}
\gantttitle{Noviembre}{30}\\
\gantttitle{Diciembre}{12}\\
\gantttitle[
title label node/.append style={below left=7pt and -3pt}
]{Día:\quad1}{1}
\gantttitlelist{2,...,132}{1} \\

\ganttgroup[progress=0]{Planificación}{1}{32} \\
\ganttbar[
name=tarea_1_1
]{\textbf{Tarea 1.1}}{1}{24} \\
\ganttbar[
name=tarea_1_2
]{\textbf{Tarea 1.2}}{25}{32} \\

\ganttgroup[progress=0]{Preparación}{33}{48} \\
\ganttbar[
name=tarea_2_1
]{\textbf{Tarea 2.1}}{33}{40} \\
\ganttbar[
name=tarea_2_2
]{\textbf{Tarea 2.2}}{41}{48} \\

\ganttgroup[progress=0]{Modelo}{49}{72} \\
\ganttbar[
name=tarea_3_1
]{\textbf{Tarea 3.1}}{49}{56} \\
\ganttbar[
name=tarea_3_2
]{\textbf{Tarea 3.2}}{57}{60} \\
\ganttbar[
name=tarea_3_3
]{\textbf{Tarea 3.3}}{61}{64} \\
\ganttbar[
name=tarea_3_4
]{\textbf{Tarea 3.4}}{65}{68} \\
\ganttbar[
name=tarea_3_5
]{\textbf{Tarea 3.5}}{65}{68} \\
\ganttbar[
name=tarea_3_6
]{\textbf{Tarea 3.6}}{69}{72} \\

\ganttgroup[progress=0]{Usuarios}{73}{96} \\
\ganttbar[
name=tarea_4_1
]{\textbf{Tarea 4.1}}{73}{76} \\
\ganttbar[
name=tarea_4_2
]{\textbf{Tarea 4.2}}{77}{84} \\
\ganttbar[
name=tarea_4_3
]{\textbf{Tarea 4.3}}{85}{92} \\
\ganttbar[
name=tarea_4_4
]{\textbf{Tarea 4.4}}{93}{96} \\

\ganttgroup[progress=0]{Nodered}{73}{84} \\
\ganttbar[
name=tarea_5_1
]{\textbf{Tarea 5.1}}{73}{84} \\

\ganttgroup[progress=0]{Config ensayos}{73}{108} \\
\ganttbar[
name=tarea_6_1
]{\textbf{Tarea 6.1}}{73}{80} \\
\ganttbar[
name=tarea_6_2
]{\textbf{Tarea 6.2}}{81}{92} \\
\ganttbar[
name=tarea_6_3
]{\textbf{Tarea 6.3}}{93}{104} \\
\ganttbar[
name=tarea_6_4
]{\textbf{Tarea 6.4}}{105}{108} \\

\ganttgroup[progress=0]{Visualización ensayos}{73}{112} \\
\ganttbar[
name=tarea_7_1
]{\textbf{Tarea 7.1}}{73}{76} \\
\ganttbar[
name=tarea_7_2
]{\textbf{Tarea 7.2}}{77}{84} \\
\ganttbar[
name=tarea_7_3
]{\textbf{Tarea 7.3}}{85}{108} \\
\ganttbar[
name=tarea_7_4
]{\textbf{Tarea 7.4}}{109}{112} \\

\ganttgroup[progress=0]{Relación con hardware}{73}{92} \\
\ganttbar[
name=tarea_8_1
]{\textbf{Tarea 8.1}}{73}{80} \\
\ganttbar[
name=tarea_8_2
]{\textbf{Tarea 8.2}}{81}{92} \\

\ganttgroup[progress=0]{Tests de aprobación}{113}{132} \\
\ganttbar[
name=tarea_9_1
]{\textbf{Tarea 9.1}}{113}{132} \\

%Links
\ganttlink{tarea_1_1}{tarea_1_2}
\ganttlink{tarea_1_2}{tarea_2_1}
\ganttlink{tarea_2_1}{tarea_2_2}

\ganttlink{tarea_2_2}{tarea_3_1}
\ganttlink{tarea_3_1}{tarea_3_2}
\ganttlink{tarea_3_2}{tarea_3_3}
\ganttlink{tarea_3_2}{tarea_3_4}
\ganttlink{tarea_3_2}{tarea_3_5}
\ganttlink{tarea_3_5}{tarea_3_6}

\ganttlink{tarea_3_6}{tarea_4_1}
\ganttlink{tarea_4_1}{tarea_4_2}
\ganttlink{tarea_4_2}{tarea_4_3}
\ganttlink{tarea_4_3}{tarea_4_4}

\ganttlink{tarea_3_6}{tarea_5_1}

\ganttlink{tarea_3_6}{tarea_6_1}

\ganttlink{tarea_3_6}{tarea_7_1}

\ganttlink{tarea_3_1}{tarea_8_1}

\ganttlink{tarea_7_4}{tarea_9_1}
\ganttlink{tarea_6_4}{tarea_9_1}
\ganttlink{tarea_4_4}{tarea_9_1}

\end{ganttchart}

}%
  \caption{Diagrama de Gantt}
  \end{figure}

\end{landscape}


\section{9. Matriz de uso de recursos de materiales}
\label{sec:recursos}


\begin{table}[htpb]
\label{tab:recursos}
\centering
\begin{tabular}{|l|l|c|c|}
\hline
\multicolumn{1}{|c|}{\multirow{2}{*}{\begin{tabular}[c]{@{}c@{}}Código\\ WBS\end{tabular}}} & \multicolumn{1}{c|}{\multirow{2}{*}{Nombre tarea}} & \multicolumn{2}{c|}{Uso de recursos [hs]} \\ \cline{3-4} 
                                                                  & \multicolumn{1}{c|}{}                              & Notebook 1 & Servidor \\ \hline

1.1 & Leer documentación del proyecto  & 48 &    \\ \hline
1.2 & Reuniones con el equipo & 16 &    \\ \hline
2.1 & Organizar herramientas de desarrollo & 16 &  8 \\ \hline
2.2 & Generación del entorno y el repositorio & 16  & 8 \\ \hline
3.1 & Definir entidades y sus relaciones & 16  &  \\ \hline
3.2 & Crear las migraciones para generar las tablas & 8  &  \\ \hline
3.3 & Crear Seeders para las tablas & 8  &  \\ \hline
3.4 & Crear servicio de acceso a base de datos & 16  &  \\ \hline
3.5 & Crear API de acceso a datos & 16  &  \\ \hline
3.6 & Verificación & 8  &  \\ \hline
4.1 & Estudiar y agregar plugin de autenticación & 8  &  \\ \hline
4.2 & Estudiar plugin de roles & 16  &  \\ \hline
4.3 & Agregar plugin de roles y definir roles & 16  &  \\ \hline
4.4 & Verificación & 8  &  \\ \hline
5.1 & Implementar recepción e inserción de registros & 24  & 24 \\ \hline
6.1 & Definir vistas de configuración & 16  &  \\ \hline
6.2 & Implementar componentes Vue & 24  &  \\ \hline
6.3 & Implementar vistas & 24  &  \\ \hline
6.4 & Verificación & 8  &  \\ \hline
7.1 & Estudiar Google Vue Charts & 8  &  \\ \hline
7.2 & Diseñar vistas para cada tipo de ensayo & 16  &  \\ \hline
7.3 & Implementar componente Vue para cada tipo & 48  &  \\ \hline
7.4 & Verificación & 8  &  \\ \hline
8.1 & Acordar estructuras de datos & 16  &  \\ \hline
8.2 & Reuniones de asistencia e intercambio & 16  &  \\ \hline
9.1 & Realizar los tests de aprobación & 32  & 32 \\ \hline
\end{tabular}
\end{table}


\section{10. Presupuesto detallado del proyecto}
\label{sec:presupuesto}

\begin{comment}
\begin{consigna}{red}
Si el proyecto es complejo entonces separarlo en partes:
\begin{itemize}
\item Un total global, indicando el subtotal acumulado por cada una de las áreas.
\item El desglose detallado del subtotal de cada una de las áreas.
\end{itemize}

IMPORTANTE: No olvidarse de considerar los COSTOS INDIRECTOS.

\end{consigna}
\end{comment}

\begin{table}[htpb]
\centering
\begin{tabularx}{\linewidth}{@{}|X|c|r|r|@{}}
\hline
\rowcolor[HTML]{C0C0C0} 
\multicolumn{4}{|c|}{\cellcolor[HTML]{C0C0C0}COSTOS DIRECTOS} \\ \hline
\rowcolor[HTML]{C0C0C0} 
Descripción &
  \multicolumn{1}{c|}{\cellcolor[HTML]{C0C0C0}Cantidad} &
  \multicolumn{1}{c|}{\cellcolor[HTML]{C0C0C0}Valor unitario} &
  \multicolumn{1}{c|}{\cellcolor[HTML]{C0C0C0}Valor total} \\ \hline
\multicolumn{1}{|l|}{Desarrollo integral de la aplicación} & 600 horas
   & \$ 1.000
   & \$ 600.000
   \\ \hline
\multicolumn{3}{|c|}{SUBTOTAL} &
  \multicolumn{1}{c|}{} \\ \hline
\rowcolor[HTML]{C0C0C0} 
\multicolumn{4}{|c|}{\cellcolor[HTML]{C0C0C0}COSTOS INDIRECTOS} \\ \hline
\rowcolor[HTML]{C0C0C0} 
Descripción &
  \multicolumn{1}{c|}{\cellcolor[HTML]{C0C0C0}Cantidad} &
  \multicolumn{1}{c|}{\cellcolor[HTML]{C0C0C0}Valor unitario} &
  \multicolumn{1}{c|}{\cellcolor[HTML]{C0C0C0}Valor total} \\ \hline
\multicolumn{1}{|l|}{10\% del costo directo por imponderables} & ---
   & ---
   & \$ 60.000
   \\ \hline
\multicolumn{3}{|c|}{SUBTOTAL} &
  \multicolumn{1}{c|}{} \\ \hline
\rowcolor[HTML]{C0C0C0}
\multicolumn{3}{|c|}{TOTAL} & \$ 660.000
   \\ \hline
\end{tabularx}%
\end{table}

\newpage
\section{11. Matriz de asignación de responsabilidades}
\label{sec:responsabilidades}

\begin{comment}
\begin{consigna}{red}
Establecer la matriz de asignación de responsabilidades y el manejo de la autoridad completando la siguiente tabla:


{\footnotesize
Referencias:
\begin{itemize}
	\item P = Responsabilidad Primaria
	\item S = Responsabilidad Secundaria
	\item A = Aprobación
	\item I = Informado
	\item C = Consultado
\end{itemize}
} %footnotesize

Una de las columnas debe ser para el Director, ya que se supone que participará en el proyecto.
A su vez se debe cuidar que no queden muchas tareas seguidas sin ``A'' o ``I''.

Importante: es redundante poner ``I/A'' o ``I/C'', porque para aprobarlo o responder consultas primero la persona debe ser informada.

\end{consigna}
\end{comment}

\begin{table}[htpb]
\centering
\resizebox{\textwidth}{!}{%
\begin{tabular}{|c|l|c|c|c|c|}
\hline
\rowcolor[HTML]{C0C0C0} 
\cellcolor[HTML]{C0C0C0} &
  \cellcolor[HTML]{C0C0C0} &
  \multicolumn{4}{c|}{\cellcolor[HTML]{C0C0C0}Listar todos los nombres y roles del proyecto} \\ \cline{3-6} 
\rowcolor[HTML]{C0C0C0} 
\cellcolor[HTML]{C0C0C0} &
  \cellcolor[HTML]{C0C0C0} &
  Responsable &
  Orientador &
  Equipo &
  Cliente \\ \cline{3-6} 
\rowcolor[HTML]{C0C0C0} 
\multirow{-3}{*}{\cellcolor[HTML]{C0C0C0}\begin{tabular}[c]{@{}c@{}}Código\\ WBS\end{tabular}} &
  \multirow{-3}{*}{\cellcolor[HTML]{C0C0C0}Nombre de la tarea} &
  \authorname &
  \supname &
  Adrian Laiuppa &
  \clientename \\ \hline
1.1 & Leer documentación del proyecto  & P & S & S & A \\ \hline
1.2 & Reuniones con el equipo & P & P & P & I \\ \hline
2.1 & Organizar herramientas de desarrollo & P & A &  & \\ \hline
2.2 & Generación del entorno y el repositorio & P & A &  &  \\ \hline
3.1 & Definir entidades y sus relaciones & P & A &  &  \\ \hline
3.2 & Crear las migraciones para generar las tablas & P & A &  &  \\ \hline
3.3 & Crear Seeders para las tablas & P & A &  &  \\ \hline
3.4 & Crear servicio de acceso a base de datos & P & A &  &  \\ \hline
3.5 & Crear API de acceso a datos & P & A &  &  \\ \hline
3.6 & Verificación & S & P & A & I \\ \hline
4.1 & Estudiar y agregar plugin de autenticación & P &  &  &  \\ \hline
4.2 & Estudiar plugin de roles & P & I & I &  \\ \hline
4.3 & Agregar plugin de roles y definir roles & P & A & I &  \\ \hline
4.4 & Verificación & S & P & A & I \\ \hline
5.1 & Implementar recepción e inserción de registros & P & A & I & \\ \hline
6.1 & Definir vistas de configuración & P & A &  &  \\ \hline
6.2 & Implementar componentes Vue & P & A &  &  \\ \hline
6.3 & Implementar vistas & P & A &  &  \\ \hline
6.4 & Verificación & S & P & A & I \\ \hline
7.1 & Estudiar Google Vue Charts & P &  &  &  \\ \hline
7.2 & Diseñar vistas para cada tipo de ensayo & P & A &  &  \\ \hline
7.3 & Implementar componente Vue para cada tipo & P & A &  &  \\ \hline
7.4 & Verificación & S & P & S &  \\ \hline
8.1 & Acordar estructuras de datos & P & S & S &  \\ \hline
8.2 & Reuniones de asistencia e intercambio & P &  &  &  \\ \hline
9.1 & Realizar los tests de aprobación & S & P & A & A \\ \hline
\end{tabular}%
}
\end{table}


\section{12. Gestión de riesgos}
\label{sec:riesgos}

\begin{comment}
\begin{consigna}{red}
a) Identificación de los riesgos (al menos cinco) y estimación de sus consecuencias:
 
Riesgo 1: detallar el riesgo (riesgo es algo que si ocurre altera los planes previstos)
\begin{itemize}
\item Severidad (S): mientras más severo, más alto es el número (usar números del 1 al 10).\\
Justificar el motivo por el cual se asigna determinado número de severidad (S).
\item Probabilidad de ocurrencia (O): mientras más probable, más alto es el número (usar del 1 al 10).\\
Justificar el motivo por el cual se asigna determinado número de (O). 
\end{itemize}   

Riesgo 2:
\begin{itemize}
\item Severidad (S): 
\item Ocurrencia (O):
\end{itemize}

Riesgo 3:
\begin{itemize}
\item Severidad (S): 
\item Ocurrencia (O):
\end{itemize}
\end{comment}


b) Tabla de gestión de riesgos:      (El RPN se calcula como RPN=SxO)

\begin{comment}
\begin{table}[htpb]
\centering
\begin{tabularx}{\linewidth}{@{}|X|c|c|c|c|c|c|@{}}
\hline
\rowcolor[HTML]{C0C0C0} 
Riesgo & S & O & RPN & S* & O* & RPN* \\ \hline
       &   &   &     &    &    &      \\ \hline
       &   &   &     &    &    &      \\ \hline
       &   &   &     &    &    &      \\ \hline
       &   &   &     &    &    &      \\ \hline
       &   &   &     &    &    &      \\ \hline
\end{tabularx}%
\end{table}

Criterio adoptado: 
Se tomarán medidas de mitigación en los riesgos cuyos números de RPN sean mayores a ....

Nota: los valores marcados con (*) en la tabla corresponden luego de haber aplicado la mitigación.

c) Plan de mitigación de los riesgos que originalmente excedían el RPN máximo establecido:
 
Riesgo 1: Plan de mitigación (si por el RPN fuera necesario elaborar un plan de mitigación).
  Nueva asignación de S y O, con su respectiva justificación:
  - Severidad (S): mientras más severo, más alto es el número (usar números del 1 al 10).
          Justificar el motivo por el cual se asigna determinado número de severidad (S).
  - Probabilidad de ocurrencia (O): mientras más probable, más alto es el número (usar del 1 al 10).
          Justificar el motivo por el cual se asigna determinado número de (O).

Riesgo 2: Plan de mitigación (si por el RPN fuera necesario elaborar un plan de mitigación).
 
Riesgo 3: Plan de mitigación (si por el RPN fuera necesario elaborar un plan de mitigación)

\end{consigna}
\end{comment}


R1: Ausencia por enfermedad o fuerza mayor, de alguno de los responsables del proyecto.
\begin{itemize}
\item Severidad (S): 10, porque puede derivar en no terminar a tiempo el proyecto.
\item Ocurrencia (O): 5, puede ocurrir pero comunmente son pocos días.
\end{itemize}

R2: Falta de tiempo en ocasiones aisladas.
\begin{itemize}
\item Severidad (S): 8, idem anterior.
\item Ocurrencia (O): 6, porque cada uno tiene sus propias actividades regulares.
\end{itemize}

R3: Pérdida del código fuente.
\begin{itemize}
\item Severidad (S): 10, significa tener que rehacer el trabajo perdido. 
\item Ocurrencia (O): 5, la falla puede ocurrir en las computadoras personales.
\end{itemize}

R4: Cambios en los requerimientos durante el desarrollo.
\begin{itemize}
\item Severidad (S): 8, puede significar una extensión de la duración del proyecto.
\item Ocurrencia (O): 5, en proyectos largos es más probable que se propongan modificaciones.
\end{itemize}

R5: Falta de coordinación con el equipo de Hardware.
\begin{itemize}
\item Severidad (S): 7, es un trabajo hecho en interdependencia.
\item Ocurrencia (O): 2, hay tiempo suficiente para coordinar acciones.
\end{itemize}


\begin{table}[htpb]
\centering
\begin{tabularx}{\linewidth}{@{}|X|c|c|c|c|c|c|@{}}
\hline
\rowcolor[HTML]{C0C0C0} 
Riesgo & S & O & RPN & S* & O* & RPN* \\ \hline
R1 & 10 & 5 & 50 & 8 & 2 & 16    \\ \hline
R2 & 8 & 6 & 48 & 6 & 4 & 24   \\ \hline
R3 & 10 & 5 & 50 & 5 & 1 & 5   \\ \hline
R4 & 8 & 5 & 40 & 8 & 2 & 16   \\ \hline
R5 & 7 & 2 & 14 &  &  &    \\ \hline
\end{tabularx}%
\end{table}

Criterio adoptado: 
Se tomarán medidas de mitigación en los riesgos cuyos números de RPN sean mayores a 25

Nota: los valores marcados con (*) en la tabla corresponden luego de haber aplicado la mitigación.

c) Plan de mitigación de los riesgos que originalmente excedían el RPN máximo establecido:

R1: Designaremos alguna persona sustituta que pueda ocupar esas horas.
\begin{itemize}
\item Severidad (S): 8, no es facil encontrar profesionales disponibles.
\item Ocurrencia (O): 2, en caso de urgencia, mi director de proyecto puede sustituirme unos días y recuperar el tiempo perdido.
\end{itemize}

R2: Recuperar tiempo los fines de semana.
\begin{itemize}
\item Severidad (S): 6, podría tener otros compromisos aislados.
\item Ocurrencia (O): 4, en general puedo dedicar más tiempo los fines de semana.
\end{itemize}

R3: Basamos la documentación en el uso de repositorios en la nube.
\begin{itemize}
\item Severidad (S): 5, si se pierde en forma local, contamos con el repositorio. 
\item Ocurrencia (O): 1, los repositorios en la nube son muy confiables.
\end{itemize}

R4: Se conversarán bien los requerimientos para minimizar este tipo de intervenciones.
\begin{itemize}
\item Severidad (S): 8, sigue significando una extensión de la duración del proyecto.
\item Ocurrencia (O): 2, puede quedar algún detalle olvidado.
\end{itemize}



\section{13. Gestión de la calidad}
\label{sec:calidad}

\begin{comment}
\begin{consigna}{red}
Para cada uno de los requerimientos del proyecto indique:

\begin{itemize}
\item Req \#1: Copiar acá el requerimiento

Verificación y validación:

\begin{itemize}
\item Verificación para confirmar si se cumplió con lo requerido antes de mostrar el sistema al cliente:\\
Detallar
\item Validación con el cliente para confirmar que está de acuerdo en que se cumplió con lo requerido:\\
Detallar
\end{itemize}

Tener en cuenta que en este contexto se pueden mencionar simulaciones, cálculos, revisión de hojas de datos, consulta con expertos, etc.

\end{consigna}
\end{comment}



\begin{itemize}
\item Req \#1: Se pedirá al usuario autenticarse mediante nombre y password para usar el sistema. El usuario no autenticado será redirigido a una pantalla de login.

Verificación y validación:

\begin{itemize}
\item Verificación para confirmar si se cumplió con lo requerido antes de mostrar el sistema al cliente:\\
Chequear en el código si la funcionalidad está implementada. Realizar un test.
\item Validación con el cliente para confirmar que está de acuerdo en que se cumplió con lo requerido:\\
Mostrar al cliente el proceso de validación de forma práctica. 
\end{itemize}

\item Req \#2: Habrá tres (3) roles de usuarios: Administrador, Configurador y Usuario de solo lectura.
		\begin{enumerate}
		\item El usuario de tipo 'Administrador' sólo será autorizado a realizar operaciones relativas a administrar usuarios.
		\item El usuario de tipo 'Configurador' sólo será autorizado a realizar operaciones relativas a configuración de ensayos.
		\item El usuario de tipo 'Solo lectura' sólo será autorizado a realizar operaciones relativas a la visualización de resultados.
		\end{enumerate}

Verificación y validación:

\begin{itemize}
\item Verificación para confirmar si se cumplió con lo requerido antes de mostrar el sistema al cliente:\\
Verificar que el código contempla los tres tipos de usario.
 
\item Validación con el cliente para confirmar que está de acuerdo en que se cumplió con lo requerido:\\
Mostrar al cliente que para tipo de usuario usando el sistema, las funciones disponibles varían y que sólo pueden realizar las acciones permitidas.
\end{itemize}


\item Req \#3: Habrá una vista para la configuración de ensayo. La vista permitirá al usuario seleccionar el relé, indicar el tipo de ensayo y los parámetros pertinentes al ensayo.

Verificación y validación:

\begin{itemize}
\item Verificación para confirmar si se cumplió con lo requerido antes de mostrar el sistema al cliente:\\
Testear la vista de configuración de ensayo.
\item Validación con el cliente para confirmar que está de acuerdo en que se cumplió con lo requerido:\\
Mostrar al cliente el proceso práctico de configuración de ensayo.
\end{itemize}


\item Req \#4: Los datos indicados para el ensayo serán validados, con indicación visual y clara para el usuario.

Verificación y validación:

\begin{itemize}
\item Verificación para confirmar si se cumplió con lo requerido antes de mostrar el sistema al cliente:\\
Chequear el módulo de validación de datos. Hacer un test.
\item Validación con el cliente para confirmar que está de acuerdo en que se cumplió con lo requerido:\\
Mostrar al cliente qué ocurre cuando se ingresan datos equivocados.
\end{itemize}


\item Req \#5: Habrá una vista del listado índice de relés. Al seleccionar un relé del listado, se mostrará una vista de resultados de ensayos del relé seleccionado.

Verificación y validación:

\begin{itemize}
\item Verificación para confirmar si se cumplió con lo requerido antes de mostrar el sistema al cliente:\\
Chequear visualmente el código fuente. Hacer una test de funcionamiento.
\item Validación con el cliente para confirmar que está de acuerdo en que se cumplió con lo requerido:\\
Mostrar al cliente la operación de la vista.
\end{itemize}


\item Req \#6: La vista de resultados de ensayos del relé mostrará los resultados de los ensayos en forma numérica y gráfica, de fácil interpretación por parte del usuario.

Verificación y validación:

\begin{itemize}
\item Verificación para confirmar si se cumplió con lo requerido antes de mostrar el sistema al cliente:\\
Chequear el código. Hacer una prueba de visualización e interpretación práctica de la vista.
\item Validación con el cliente para confirmar que está de acuerdo en que se cumplió con lo requerido:\\
Mostrar al cliente la vista, e indagar sobre el entendimiento de la misma.
\end{itemize}


\item Req \#7: El sistema deberá ser capaz de comunicarse con potencialmente más de un equipo probador de relés. Inicialmente será un único equipo, pero este número puede ser mayor en el futuro.

Verificación y validación:

\begin{itemize}
\item Verificación para confirmar si se cumplió con lo requerido antes de mostrar el sistema al cliente:\\
Examinar la lógica de comunicación en la implementación y verificar que el ID de hardware participa como discriminador.
\item Validación con el cliente para confirmar que está de acuerdo en que se cumplió con lo requerido:\\
Mostrar al cliente una prueba cambiando el ID del hardware y verificando que se logra la comunicación como si se tratara de otro equipo.
\end{itemize}


\item Req \#8: Al seleccionar un relé del listado, se mostrará una pantalla de resultados de ensayos del relé.

Verificación y validación:

\begin{itemize}
\item Verificación para confirmar si se cumplió con lo requerido antes de mostrar el sistema al cliente:\\
Chequear el comportamiento esperado.
\item Validación con el cliente para confirmar que está de acuerdo en que se cumplió con lo requerido:\\
Mostrar al cliente el comportamiento.
\end{itemize}


\item Req \#9: Los datos serán persistidos en un servidor y base de datos provista por el cliente.

Verificación y validación:

\begin{itemize}
\item Verificación para confirmar si se cumplió con lo requerido antes de mostrar el sistema al cliente:\\
Chequear la base de datos. Realizar ensayos y comprobar que los datos recibidos son guardados correctamente.
\item Validación con el cliente para confirmar que está de acuerdo en que se cumplió con lo requerido:\\
Mostrar al cliente una simulación de ensayo recibido y cómo estos se guardan en la base de datos correcta.
\end{itemize}


\item Req \#10: Cualquier registro de la base de datos, tendrá referencia al último usuario que lo modificó, junto con la fecha.

Verificación y validación:

\begin{itemize}
\item Verificación para confirmar si se cumplió con lo requerido antes de mostrar el sistema al cliente:\\
Chequear la implementación del código.
\item Validación con el cliente para confirmar que está de acuerdo en que se cumplió con lo requerido:\\
Mostrar al cliente que se puede ver el usuario y fecha en un registro modificado.
\end{itemize}

\end{itemize}


\section{14. Comunicación del proyecto}
\label{sec:comunicaciones}

\begin{comment}
\begin{consigna}{red}
El plan de comunicación del proyecto es el siguiente:
\end{consigna}
\end{comment}

% Please add the following required packages to your document preamble:
% \usepackage{graphicx}
% \usepackage[table,xcdraw]{xcolor}
% If you use beamer only pass "xcolor=table" option, i.e. \documentclass[xcolor=table]{beamer}
\begin{table}[htpb]
\centering
\resizebox{\textwidth}{!}{%
\begin{tabular}{|l|c|c|c|c|c|}
\hline
\rowcolor[HTML]{C0C0C0} 
\multicolumn{6}{|c|}{\cellcolor[HTML]{C0C0C0}PLAN DE COMUNICACIÓN DEL PROYECTO}           \\ \hline
\rowcolor[HTML]{C0C0C0} 
¿Qué comunicar? & Audiencia & Propósito & Frecuencia & Método de comunicac. & Responsable \\ \hline
Avances en el diseño & Equipo & Informar & Semanal & Email & \authorname \\ \hline
Avances en la implementación & Equipo & Informar, revisar & Semanal & Reunión & \authorname \\ \hline
Imprevistos & Todos & Resolver, acordar & Cuando ocurra & Reunión & \supname \\ \hline
\end{tabular}%
}
\end{table}


\section{15. Gestión de Compras}
\label{sec:compras}

\begin{comment}
\begin{consigna}{red}
En caso de tener que comprar elementos o contratar servicios:
a) Explique con qué criterios elegiría a un proveedor.
b) Redacte el Statement of Work correspondiente.
\end{consigna}
\end{comment}

No hay compras involucradas

\newpage

\section{16. Seguimiento y control}
\label{sec:seguimiento}

\begin{comment}
\begin{consigna}{red}
Para cada tarea del proyecto establecer la frecuencia y los indicadores con los se seguirá su avance y quién será el responsable de hacer dicho seguimiento y a quién debe comunicarse la situación (en concordancia con el Plan de Comunicación del proyecto).

El indicador de avance tiene que ser algo medible, mejor incluso si se puede medir en \% de avance. Por ejemplo,se pueden indicar en esta columna cosas como ``cantidad de conexiones ruteadeas'' o ``cantidad de funciones implementadas'', pero no algo genérico y ambiguo como ``\%'', porque el lector no sabe porcentaje de qué cosa.

\end{consigna}
\end{comment}

\begin{table}[!htpb]
\centering
\begin{tabularx}{\linewidth}{@{}|c|X|X|X|X|X|@{}}
\hline
\rowcolor[HTML]{C0C0C0} 
\multicolumn{6}{|c|}{\cellcolor[HTML]{C0C0C0}SEGUIMIENTO DE AVANCE}                                                                       \\ \hline
\rowcolor[HTML]{C0C0C0} 
Tarea del WBS & Indicador de avance & Frecuencia de reporte & Resp. de seguimiento & Persona a ser informada & Método de comunic. \\ \hline
1.1 & Avance documento & Semanal & \supname & \clientename & Planilla de avance \\ \hline
1.2 & Reuniones realizadas & Semanal & \supname & \clientename & Planilla de avance \\ \hline
2.1 & Pendiente / Listo & Semanal & \supname & \clientename & Planilla de avance  \\ \hline
2.2 & Pendiente / Listo & Semanal & \supname & \clientename & Planilla de avance  \\ \hline
3.1 & Pendiente / En progreso / Listo & Semanal & \supname & \clientename & Planilla de avance  \\ \hline
3.2 & Migraciones implementadas & Semanal & \supname & \clientename & Planilla de avance  \\ \hline
3.3 & Seeders implementados & Semanal & \supname & \clientename & Planilla de avance  \\ \hline
3.4 & Funciones implementadas & Semanal & \supname & \clientename & Planilla de avance  \\ \hline
3.5 & Funciones implementadas & Semanal & \supname & \clientename & Planilla de avance  \\ \hline
3.6 & Pendiente / Listo & Semanal & \supname & \clientename & Planilla de avance  \\ \hline
4.1 & Pendiente / Listo & Semanal & \supname & \clientename & Planilla de avance  \\ \hline
4.2 & Pendiente / Listo & Semanal & \supname & \clientename & Planilla de avance  \\ \hline
4.3 & Pendiente / En progreso / Listo & Semanal & \supname & \clientename & Planilla de avance  \\ \hline
4.4 & Pendiente / Listo & Semanal & \supname & \clientename & Planilla de avance  \\ \hline
5.1 & Pendiente / En progreso / Listo & Semanal & \supname & \clientename & Planilla de avance  \\ \hline
6.1 & Vistas definidas & Semanal & \supname & \clientename & Planilla de avance  \\ \hline
6.2 & Componentes implementados & Semanal & \supname & \clientename & Planilla de avance  \\ \hline
\end{tabularx}%
%}
\end{table}

\begin{table}[!htpb]
\centering
\begin{tabularx}{\linewidth}{@{}|c|X|X|X|X|X|@{}}
\hline
\rowcolor[HTML]{C0C0C0} 
\multicolumn{6}{|c|}{\cellcolor[HTML]{C0C0C0}SEGUIMIENTO DE AVANCE}                                                                       \\ \hline
\rowcolor[HTML]{C0C0C0} 
Tarea del WBS & Indicador de avance & Frecuencia de reporte & Resp. de seguimiento & Persona a ser informada & Método de comunic. \\ \hline
6.3 & Vistas implementadas & Semanal & \supname & \clientename & Planilla de avance  \\ \hline
6.4 & Pendiente / Listo & Semanal & \supname & \clientename & Planilla de avance  \\ \hline
7.1 & Pendiente / Listo & Semanal & \supname & \clientename & Planilla de avance  \\ \hline
7.2 & Vistas diseñadas & Semanal & \supname & \clientename & Planilla de avance  \\ \hline
7.3 & Componentes implementados & Semanal & \supname & \clientename & Planilla de avance  \\ \hline
7.4 & Pendiente / Listo & Semanal & \supname & \clientename & Planilla de avance  \\ \hline
8.1 & Pendiente / Listo & Semanal & \supname & \clientename & Planilla de avance  \\ \hline
8.2 & Reuniones realizadas & Semanal & \supname & \clientename & Planilla de avance  \\ \hline
9.1 & Tests realizados & Semanal & \supname & \clientename & Reporte  \\ \hline
\end{tabularx}%
\end{table}

\newpage

\section{17. Procesos de cierre}    
\label{sec:cierre}

\begin{comment}
\begin{consigna}{red}
Establecer las pautas de trabajo para realizar una reunión final de evaluación del proyecto, tal que contemple las siguientes actividades:

\begin{itemize}
\item Pautas de trabajo que se seguirán para analizar si se respetó el Plan de Proyecto original:
 - Indicar quién se ocupará de hacer esto y cuál será el procedimiento a aplicar. 
\item Identificación de las técnicas y procedimientos útiles e inútiles que se utilizaron, y los problemas que surgieron y cómo se solucionaron:
 - Indicar quién se ocupará de hacer esto y cuál será el procedimiento para dejar registro.
\item Indicar quién organizará el acto de agradecimiento a todos los interesados, y en especial al equipo de trabajo y colaboradores:
  - Indicar esto y quién financiará los gastos correspondientes.
\end{itemize}

\end{consigna}
\end{comment}

Pautas de trabajo que se seguirán para analizar si se respetó el Plan de Proyecto original:

\begin{itemize}

\item Reunión previa con el equipo para organizar el cierre.

\item Evaluar los resultados del proceso. Evaluar el grado de cumplimiento de la planificación inicial.
\begin{enumerate}
\item Grado de cumplimiento de los requerimientos.
\item Grado de cumplimiento del Gantt.
\item Grado de cumplimiento la gestión de riesgos.
\item Grado de cumplimiento la calidad alcanzada.
\item Grado de satisfacción del cliente.
\end{enumerate}

\item Acto de agradecimiento a todos los interesados y participantes del proyecto.

\end{itemize}


Responsable: \authorname

\end{document}
